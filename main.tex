\documentclass{article}
\usepackage[utf8]{inputenc}
\usepackage{graphicx}

\title{Reporte Técnico Parcial 2 Métodos Numérico}
\author{A01026400, A01733222, A01228334, A01327811, A01351413 }
\date{November 2021}

\begin{document}

\maketitle

\section{Introducción}
\itemEn este proyecto de segundo parcial de la materia de métodos numéricos hemos decidido utilizar el método de Cramer y el de Gauss-Jordan, de ecuaciones lineales para resolver un problema común al que nos enfrentaremos al tener nuestro propio negocio.

\section{Descripción del problema a resolver}
\item Imaginemos que tenemos un taller automovilístico, y mensualmente se hacen de diferentes partes y herramientas que casi cualquier automóvil necesita, es decir, nada muy específico. Nosotros como jefes del taller no estamos muy al tanto de qué es lo que se compra (ni de cantidades ni de precios) muy explícitamente, es decir, sabemos del presupuesto requerido y qué tipo de martes nada más. Sabemos que se gastaron 75,000 el último mes entre lámparas (linternas), tuercas, y líquido refrigerante. Las lámparas cuestan 160, los paquetes que traen diferentes tuercas 800 y las cajas de 10 líquidos cuestan 500. El número de la caja de líquidos compradas es igual al número de tuercas + número de lámparas compradas. El método que usaremos para saber cuánto se compró de cada cosa es el de Cramer. 

La primera herramienta que utilizaremos es Excel, ya que ahí haremos las matrices y resolveremos este problema. Después correremos el código en Matlab para comprobar que nuestros resultados sean correctos. 

\section{Resultados}
\item Número de lámparas: x
\item Número de cajas de líquido refrigerante: y 
\item Número de tuercas: z


\item Sistema de Ecuaciones: 
\item Primera ecuación: si sabemos que entre los 3 diferentes materiales son 200 unidades. 
\item x + y + z = 200
\item Segunda ecuación: si sabemos que se gastaron 75,000. 
\item 160x + 500y + 800z = 75,000
\item Tercera ecuación: Caja de líquidos = tuercas + lámparas
\item x = y+ z
\item x -y-z = 0

\subsection{Método de Cramer}


    
    \includegraphics[scale=0.4]{cramer}
    \caption{Excel Método de Cramer}
    \label{fig:my_label}
    
        \includegraphics[scale=0.4]{crammat.png}
    \caption{Código Método de Cramer}
    \label{fig:my_label}
    
    

\subsection{Método Gauss-Jordan}
    \includegraphics[scale=0.4]{excelgauss.png}
    \caption{Excel Método Gauss-Jordan}
    \label{fig:my_label}
    
        \includegraphics[scale=0.4]{Captura de pantalla (123).png}
    \caption{Código Método Gauss-Jordan}
    \label{fig:my_label}



\section{Conclusiones}
\item Gracias a los 2 diferentes métodos podemos darnos cuenta de que no importa la manera en la que lo resolvamos, sino que sea correcta y los datos estén acomodados bien, y nos darán el mismo resultado. Dicho esto, y habiendo resuelto, podemos decir que se compraron 100 lámparas, 70 paquetes de tuercas y 30 cajas de 10 líquidos refrigerantes. Aunque sabemos que en la vida real son muchas unidades, estamos imaginando que sea un taller muy grande que atienda a muchos coches al mes y por eso se necesita esa cantidad de material. 



\end{document}
